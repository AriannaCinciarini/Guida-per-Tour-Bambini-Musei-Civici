\documentclass[hidelinks,12pt,a4paper]{exam}
\usepackage[italian]{babel}
\usepackage[utf8]{inputenc}
\usepackage{fourier} 

% To clone parts of text
\usepackage{clipboard}

% For Creating a table
\usepackage{tabularx}

% Adjust paragraph.
\usepackage{changepage}

% License
\usepackage[
type={CC},
modifier={by-nc-sa},
version={4.0},
]{doclicense}

% Command to create fake sections
\newcommand{\fakesection}[1]{%
	\par\refstepcounter{section}% Increase section counter
	\sectionmark{#1}% Add section mark (header)
	\addcontentsline{toc}{section}{\protect\numberline{\thesection}#1}% Add section to ToC
	% Add more content here, if needed.
}

% Create a special enumerate enviroment
\usepackage{enumitem}% http://ctan.org/pkg/enumitem
\newlist{enumAnswers}{enumerate}{1}
\setlist[enumAnswers,1]{label=\textbf{Risposta domanda \arabic*. }}

\begin{document}
	
	\title{\textbf{Test per Bambini}}
	\author{Alice Balestieri\\Francesco Rombaldoni}
	\date{}
	\maketitle
	
	\newpage
	\pagestyle{plain}
	\tableofcontents
	\newpage
	
	\section{Come somministrare e correggere il Test}
	\begin{center}
		\textbf{I consigli sono rivolti agli operatori.}
	\end{center}
	
	Questo test serve per valutare la comprensione del tour guidato ai Musei Civici da parte dei bambini, per questo motivo non è necessario somministrare il compito ai singoli bambini, ma per favorire il lavoro di gruppo è consigliabile di dividere i bambini in gruppi di massimo tre elementi (in questo caso il campo "nome" diventa il nome del gruppo, oppure i nomi dei componenti della squadra). Il tempo consigliato per completare il test è di circa quindici minuti.\\
	Scaduto il tempo raccogliere i compiti ed iniziare la correzione, per questa fase si consiglia in particolare di segnare con una penna rossa gli errori commessi e con la stessa scrivere nel campo "voto" la valutazione del test. Per segnare il voto si è preferibile usare una valutazione descrittiva (come la scala: sufficiente, discreto, buono, distinto, ottimo) piuttosto che una valutazione numerica (esempio quella decimale), in modo che i bambini non possano fare grandi confronti tra di loro, in modo da tenere un clima più sereno anche al fronte di un voto che potrebbe creare disparità.\\
	Le ultime tre domande aperte non sono da correggere, siccome servono per capire le preferenze dei bambini.
	
	\begin{center}
		\large{\textbf{Tabella per la correzione}}\\
		\bigskip
		
		\begin{tabularx}{0.5\textwidth} { 
				| >{\raggedright\arraybackslash}X 
				| >{\centering\arraybackslash}X | }
			\hline
			\textbf{Descrizione} & \textbf{Voto} \\
			\hline
			Sono stati commessi massimo due errori & Ottimo\\
			\hline
			Sono stati commessi dai tre ai cinque errori & Distinto\\
			\hline
			Sono stati commessi dai sei agli otto errori & Buono\\
			\hline
			Sono stati commessi dai nove ai dieci errori & Discreto\\
			\hline
			Sono stati commessi numerosi errori & Sufficiente\\
			\hline
		\end{tabularx}
	\end{center}
	
	\newpage
	
	\fakesection{Test}
	% Remove page numbers
	\pagestyle{empty}
	
	\fboxrule=2pt
	\centerline{
		\fbox{
			\begin{minipage}{\linewidth}
				\centering{\textbf{Nome:}\line(1,0){200}}\\
				\textbf{Data:}\line(1,0){50}
				\hfill
				\textbf{Voto:}\line(1,0){50}
			\end{minipage}
		}
	}
	
	% Starting Multiple Choice Questions here
	\begin{questions}
		\Copy{questions}
		{
			% This is a question
			\question Which of these famous physicists published a paper on Brownian Motion?
			\begin{checkboxes}
				\choice Stephen Hawking 
				\CorrectChoice Albert Einstein
				\choice Emmy Noether
				\choice I don't know
			\end{checkboxes}
			% End of question
			
		}
	\end{questions}

	% End Multiple Choice Questions
	
	% Starting Open-Ended Questions here
	
	\begin{enumerate}[start=2] %This is a simple solution to have valid question numbers
		% This is a question
		\item "Domanda 1"\\
					\line(1,0){\linewidth}\\
					\line(1,0){\linewidth}\\
		% End of question
		\item "Domanda 2"\\
					\line(1,0){\linewidth}\\
					\line(1,0){\linewidth}\\
	\end{enumerate}
	% End Open-Ended Questions
	
	\newpage
	\section{Risposte Test}
	
	\pagestyle{plain}
	
	% Multiple Choice Questions Answers (auto-generated)
	\begin{questions}
		\printanswers
		\Paste{questions}
	\end{questions}

	% Open-ended Questions Answers
	\begin{adjustwidth}{35mm}{}
		\begin{enumAnswers}[start=2]
			\item "Risposta domanda 1"
			\item "Risposta domanda 2"
		\end{enumAnswers}
	\end{adjustwidth}
	
	\vspace*{\fill}
	% Print license shield
	\doclicenseThis
\end{document}