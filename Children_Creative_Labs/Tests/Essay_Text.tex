\documentclass[hidelinks,12pt,a4paper]{exam}
\usepackage[italian]{babel}
\usepackage[utf8]{inputenc}
\usepackage{fourier} 

% For Creating a table
\usepackage{tabularx}

% Adjust paragraph.
\usepackage{changepage}

% License
\usepackage[
type={CC},
modifier={by-nc-sa},
version={4.0},
]{doclicense}

% Command to create fake sections
\newcommand{\fakesection}[1]{%
	\par\refstepcounter{section}% Increase section counter
	\sectionmark{#1}% Add section mark (header)
	\addcontentsline{toc}{section}{\protect\numberline{\thesection}#1}% Add section to ToC
	% Add more content here, if needed.
}

% Create a special enumerate enviroment
\usepackage{enumitem}% http://ctan.org/pkg/enumitem
\newlist{enumAnswers}{enumerate}{1}
\setlist[enumAnswers,1]{label=\textbf{Risposta domanda \arabic*. }}

\begin{document}
	
	\title{\textbf{Test a risposta aperta per i bambini}}
	\author{Alice Balestieri\\Francesco Rombaldoni}
	\date{}
	\maketitle
	
	\newpage
	
	\tableofcontents
	\newpage
	
	\section{Come somministrare e correggere il Test}
	\begin{center}
		\textbf{I consigli sono rivolti agli operatori.}
	\end{center}
	
	Questo test serve per valutare la comprensione del tour guidato ai Musei Civici da parte dei bambini, per questo motivo non è necessario somministrare il compito ai singoli bambini, ma per favorire il lavoro di gruppo è consigliabile di dividere i bambini in gruppi di massimo tre elementi (in questo caso il campo "nome" diventa il nome del gruppo, oppure i nomi dei componenti della squadra). Il tempo consigliato per completare il test è di circa un'ora.\\
	Scaduto il tempo raccogliere i compiti ed iniziare la correzione, per questa fase si consiglia in particolare di segnare con una penna rossa gli errori commessi e con la stessa scrivere nel campo "voto" la valutazione del test. Per segnare il voto si è preferibile usare una valutazione descrittiva (come la scala: sufficiente, discreto, buono, distinto, ottimo) piuttosto che una valutazione numerica (esempio quella decimale), in modo che i bambini non possano fare grandi confronti tra di loro, in modo da tenere un clima più sereno anche al fronte di un voto che potrebbe creare disparità.\\
	
	\begin{center}
		\large{\textbf{Tabella per la correzione}}\\
		\bigskip
		
		\begin{tabularx}{0.5\textwidth} { 
				| >{\raggedright\arraybackslash}X 
				| >{\centering\arraybackslash}X | }
			\hline
			\textbf{Descrizione} & \textbf{Voto} \\
			\hline
			I concetti sono stati capiti perfettamente & Ottimo\\
			\hline
			Nelle risposte appaiono delle piccole imprecisioni & Distinto\\
			\hline
			Ci sono delle incertezze & Buono\\
			\hline
			Ci sono tante incertezze & Discreto\\
			\hline
			Sono stati commessi numerosi errori & Sufficiente\\
			\hline
		\end{tabularx}
	\end{center}
	
	\newpage
	
	\fakesection{Test}
	% Remove page numbers
	\pagestyle{empty}
	
	\fboxrule=2pt
	\centerline{
		\fbox{
			\begin{minipage}{\linewidth}
				\centering{\textbf{Nome:}\line(1,0){200}}\\
				\textbf{Data:}\line(1,0){50}
				\hfill
				\textbf{Voto:}\line(1,0){50}
			\end{minipage}
		}
	}
	
	% Starting questions here
	\bigskip
	\begin{enumerate}
		\item "Domanda 1"\\
					\line(1,0){\linewidth}\\
					\line(1,0){\linewidth}\\
		\item "Domanda 2"\\
					\line(1,0){\linewidth}\\
					\line(1,0){\linewidth}\\
	\end{enumerate}
	
	\newpage
	\section{Risposte Test}
	
	\pagestyle{plain}
	\begin{adjustwidth}{35mm}{}
		\begin{enumAnswers}
			\item "Risposta domanda 1"
			\item "Risposta domanda 2"
		\end{enumAnswers}
	\end{adjustwidth}
	
	\vspace*{\fill}
	% Print license shield
	\doclicenseThis
\end{document}