\documentclass[hidelinks,12pt,a4paper]{article}
\usepackage[italian]{babel}
\usepackage[utf8]{inputenc}
\usepackage{fourier} 

% Images
\usepackage{graphicx}
\usepackage{caption}
\usepackage{subcaption}
\usepackage{float}
\graphicspath{ {../Images} }

% Stop hyphenation
\usepackage[none]{hyphenat}

% Dotted frame.
\usepackage{tikz}

% For testing purpose
\usepackage{microtype}

% License
\usepackage[
type={CC},
modifier={by-nc-sa},
version={4.0},
]{doclicense}

\begin{document}
	
	\title{\textbf{\centering{Laboratorio creativo per bambini}\\Trova le opere dentro il museo.}}
	\author{Alice Balestieri}
	\date{}
	
	\maketitle
	\newpage
	
	\tableofcontents
	\newpage
	
	\section{Come giocare}
	\begin{center}
		\textbf{Le regole sono rivolte agli operatori.}
	\end{center}
	
	\subsection{Variante 1}
	Dopo aver ritagliato le immagini, consegnarle a rotazione ai bambini fino ad esaurire il mazzo, aggiungendo qualora ne fossero sprovvisti, delle penne con le quali poter scrivere sulla carta.\\
	A partire dal suggerimento fornito dalla porzione d'immagine che compone la carta, i bambini dovranno girare liberamente nel museo per ritrovare l'opera, e ricopiare la descrizione della suddetta nel riquadro posto sotto la porzione d'immagine.\\
	Quando tutti i bambini hanno finito di completare il compito, procedere con la correzione spiegata, concentrandosi in caso di errore sulle differenze tra l'opera da individuare e quella riportata dal bambino.
	
	\subsection{Variante 2}
	Dopo aver ritagliato le immagini, consegnarle a rotazione ai bambini fino ad esaurire il mazzo, aggiungendo qualora ne fossero sprovvisti, delle penne con le quali poter scrivere sulla carta.\\
	A partire dal suggerimento fornito dalla porzione d'immagine che compone la carta, i bambini dovranno esercitare la propria memoria cercando d'individuare l'opera a partire dal suggerimento e scrivendo nel riquadro sottostante quante più informazioni possibili sull'opera.\\
	Far scrivere a tutti i bambini i loro nomi su una estremità della carte possedute e successivamente raccogliere tutte le carte per procedere con la correzione spiegata.\\
	 Muovendosi nel museo, raggiungere assieme i bambini la posizione delle varie opere, e per ogni opera, raggruppare i bambini che hanno dovuto descriverla (i loro nomi sono scritti sulle carte), leggere quello che hanno scritto, per poi fare il confronto con la suddetta.
	
	
	\vspace*{\fill}
	\centering
	\fboxrule=2pt
	\fbox
	{
		\begin{minipage}{\linewidth}
			In caso di dubbi per la correzione, tenere una copia digitale di questo documento consultabile dallo "smartphone". Nella sezione "Immagini e didascalie" ogni immagine frammentata è presentata con la relativa didascalia posta inferiormente.
		\end{minipage}
	}

	\newpage
	\section{Immagini e didascalie}
	
	%---------- Begin page ----------
	
	%----- Mini page 1
	\begin{minipage}{\linewidth}
		
	
	\fbox{
	\begin{minipage} [l] [\dimexpr 0.300\textwidth \relax] [t] {\dimexpr .495\textwidth \relax}
		bla bla bla
	\end{minipage}
		
	}
	\hfill{
	\fbox{
		\begin{minipage} [r] [\dimexpr 0.300\textwidth \relax] [t] {\dimexpr .495\textwidth \relax}
			bla bla bla
		\end{minipage}
	}
}
	\end{minipage}

\begin{minipage}{\linewidth}
	
	
	\fbox{
		\begin{minipage} [l] [\dimexpr 0.300\textwidth \relax] [t] {\dimexpr .495\textwidth \relax}
			bla bla bla
		\end{minipage}
		
	}
	\hfill{
	\fbox{
		\begin{minipage} [r] [\dimexpr 0.300\textwidth \relax] [t] {\dimexpr .495\textwidth \relax}
			bla bla bla
		\end{minipage}
	}
}
\end{minipage}
	
	\vspace*{\fill}
	\centering
	\fboxrule=2pt
	\fbox
	{
		\begin{minipage}{\linewidth}
		Esempio di una didascalia che viene inserita in fondo alla pagina.
		\end{minipage}
	}
	
	%---------- End page ----------
\end{document}	